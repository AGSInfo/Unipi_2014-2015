\documentclass[portait]{article}
\author{Alessandro Sieni, Gianluca Mondini}
\title{Appunti Fisica}
\date{\today}
\usepackage{amsfonts}
\usepackage[utf8]{inputenc}
\usepackage{multicol}

\begin{document}
\maketitle
\newpage
\tableofcontents
\newpage
\section{Appunti 03 marzo 2015}
\subsection{Ordini di grandezza}
Quando si parla di ordini di grandezza intendiamo non una precisa quantità ma un'indicazione utile ad effettuare delle stime che non devono essere necessariamente precise.
\subsection{Calcolo dimensionale}
\subsubsection{Definizione}
Il calcolo dimensionale ci permette di lavorare solo con le dimensioni che compongono le componenti da studiare e ci permette di verificare se alcuni procedimenti o calcoli sono corretti dal punto di vista dimensionale, ovvero se la dimensione ottenuta è coerente con quella di ciò che dobbiamo calcolare.
Nel caso di alcuni casi semplici è possibile ottenere delle formule senza alcun procedimento, ma solo mediante l'utilizzo del caso dimensionale.
\subsubsection{Esempi}
Esaminiamo in questo esempio la formula per ottenere il tempo di caduta di un corpo da una fissata altezza.
Solo studiando in maniera puramente intuitiva il fenomeno una persona può ipotizzare che la velocità possa dipendere dall'accelerazione di gravita( g ) dal tempo di caduta (t) e dall'altezza dalla quale il corpo viene lanciato, quindi riportando queste tre grandezze sotto forma di dimensioni otteniamo che:
$$g = [V][T]^-1 = [L][T]^-2$$
e dato che $[V] = [L][T]^-1$ dobbiamo riottenere la stessa dimensione, e per l farlo dobbiamo trovare gli esponenti $x,y,z$ da assegnare a g,t,h che si andranno a moltiplicare, che ci permettano di ritrovare la stessa dimensione, in poche parole :
$$[L][T]^-1 = \underbrace{([L][T]^-2)^x}_g \underbrace{[T]^y}_t \underbrace{[L]^z}_h$$
Da questo procediamo in modo algebrico ottenendo
$$[L][T]^-1 = \underbrace{[L]^x[T]^{-2x}}_g \underbrace{[T]^y}_t \underbrace{[L]^z}_h$$
da cui abbiamo un sistema a due equazioni ma a tre variabili (x,y,z) in quanto
$$
\left \{ \begin{array}{lr}
   x+z = 1 \\
   y - 2x = -1
  \end{array} \right.
$$
Per risolvere questo problema proviamo a non considerare l'altezza come fattore determinante per calcolare la velocità e troviamo quindi come si comporta il tempo di caduta rispetto a g e h.
$$[T] = [T]^y ([L][T]^-2)^x$$
ottenendo stavolta un sistema a due equazioni e due incognite che ha come soluzione $x = -\frac{1}{2}$ e $y = \frac{1}{2}$. Infatti la formula del tempo di caduta di un corpo è proprio $\sqrt{\frac{h}{g}}$-. \\
\subsection{Vettori}
\hspace*{0.1cm}
\subsubsection{Introduzione}
\hspace*{0.1cm} \\ 
In fisica l'utilizzo di vettori invece che di semplici scalari è fondamentale in quanto ci permettono di rappresentare in modo corretto la realtà (ad esempio non è possibile rappresentare la posizione di un corpo con un solo numero, ma ne sono necessari 3, uno per ogni asse cartesiano x,y,z).
\subsubsection{Operazioni sui vettori}
Le operazioni che è possibile effettuare sui vettori sono le seguenti :
\begin{itemize}
\item Somma
\item Prodotto tra un vettore ed uno scalare
\item Prodotto scalare tra due vettori
\item Prodotto vettoriale
\end{itemize}
\textbf{\textit{Somma}} \\  \\
La somma tra due vettori restituisce un vettore le cui componenti corrispondono alla sommatoria delle componenti relativi alla solita posizione, esempio :
$$(1,2,3) + (7,-3,4) = (1+ 7,2 + (-3),3+4) = (8,-1.7)$$ \\  
\textbf{\textit{Prodotto di un vettore per uno scalare}} \\ \\
Il prodotto di un vettore per uno scalare si ottiene moltiplicando ciascuna componente per lo scalare, esempio :
$$3 * (4,2,5) = (4 * 3, 2 * 3, 5* 3) = (12,6,15)$$ \\ 
\textbf{\textit{Prodotto scalare tra due vettori}} \\ \\
Il risultato del prodotto scalare tra due vettori sarà appunto uno scalare che corrisponderà alla sommatoria del prodotto delle relative componenti dei due vettori, esempio presi $U = (1,2,3)$ e $V=(3,4,5)$ il prodotto scalare tra U e V sarà :
$$U*V = \sum_{i = 1}^3 (U_i * V_I) = (1* 3) + (2*4) + (5 * 3) = 26$$ \\ 
\textbf{\textit{Prodotto vettoriale tra due vettori}} \\ \\
\subsubsection{Rappresentazione di un vettore}
\hspace*{5cm} \\
Per rappresentare un vettore ci sono 4 possibili modi : 
\begin{itemize}
\item Mediante le coordinate
\item Mediante una freccia nel piano
\item Mediante il modulo del vettore e il suo argomento
\item Mediante l'utilizzo di versori
\end{itemize}
\section{Appunti 04 marzo 2015}
\subsection{Rappresentazione di un vettore}
\hspace*{5cm} \\
Per rappresentare un vettore ci sono 4 possibili modi : 
\begin{itemize}
\item Mediante le coordinate
\item Mediante una freccia nel piano
\item Mediante il modulo del vettore e il suo argomento
\item Mediante l'utilizzo di versori
\end{itemize}
\subsubsection{Coordinate}
Quando si rappresenta un vettore mediante coordinate si esplicita una tupla di n numeri l'esempio più classico è $V = (1,2,3)$ \newpage
\subsubsection{Freccia nel piano}
Quando si rappresenta un vettore con una freccia si utilizzano le stesse coordinate illustrate sopra ma le si proiettano in un ipotetico spazio n-dimensionale (ovviamente questo nei modelli matematici, perché in quelli fisici non si supera la terza dimensione, eccetto alcuni rari casi) e si traccia una freccia che parte dall'origine degli assi e che arriva proprio nel punto dello spazio avente come coordinate cartesiane le componenti del vettore. \\ 
\subsubsection{Modulo e Argomento}
Quando si rappresenta un vettore con la notazione modulo e argomento intendiamo esprimere il modulo del vettore (nel caso sia rappresentabile in uno spazio con una freccia la sua lunghezza partendo dall'origine degli assi) e dato che la sola lunghezza non è sufficiente (in quanto dando solo la lunghezza di una freccia si potrebbe intendere una circonferenza di centro 0 e raggio uguale al modulo della freccia) viene espresso anche l'angolo che forma il vettore con un asse (nel caso sia su un piano) e con 2 assi (nel caso sia su uno spazio tridimensionale), esempio:
$$\overrightarrow{U} = (3,30^\circ)$$
In questo esempio viene indicato che il vettore $\overrightarrow{U}$ ha un modulo dal valore 3 e un inclinazione con gli assi di $30^\circ$. \\ 
\subsubsection{Versori}
Prima di illustrare la rappresentazione per versori è opportuno esprimere il concetto di versore. \\
Il versore è un normale vettore che però rispetta queste due caratteristiche :
\begin{itemize}
\item Ha modulo pari ad 1
\item Corrisponde con uno degli assi cartesiano
\end{itemize}
Introdotte queste due caratteristiche si può subito notare l'esistenza di tre versori, corrispondenti agli assi x,y,z. Questi versori sono :
\begin{itemize}
\item $\overrightarrow{i} = (1,0,0)$
\item $\overrightarrow{j} = (0,1,0) $
\item $\overrightarrow{k} =(0,0,1),$
\end{itemize}
Costruire un qualunque vettore con l'utilizzo di questi tre versori è molto semplice in quanto basterà esprimere il vettore come una sommatori di multipli di versori, ad esempio :
$$\overrightarrow{U} = (7,-3.5 ) = 7\overrightarrow{i} + (-3)\overrightarrow{j} + 5\overrightarrow{k} =(7,0,0) + (0,-3,0) + (0,0,5 ) = (7,-3,5)$$ \\ 
\subsection{Prodotto scalare}
Un' operazione molto importante per i vettori è il prodotto scalare (il cui metodo di "elaborazione", per la rappresentazione in coordinate,è illustrato nella sezione del 04/03/2015 alla voce "operazione sui vettori") che quando siamo in rappresentazione per modulo è argomento funziona cosi (presi $\overrightarrow{U} = (\rho_1,\theta_1)$ e $\overrightarrow{V} = (\rho_2,\theta_2)$) :
$$\overrightarrow{U} * \overrightarrow{V} = |\rho_1||\rho_2| * \cos(\theta_1 - \theta_2)$$ \\ 
Da quest'ultima formula si nota subito una delle cose fondamentali del prodotto scalare, ovvero che esprime al suo interno l'angolo che si forma tra i due vettori, espresso sotto la funzione cos. Ed è proprio dalla funzione cos che si ricava subito un'informazione importante sul prodotto scalare, ovvero che se due vettori sono ortogonali (ovvero formano un angolo di 90 gradi) il loro prodotto scalare sarà uguale a zero, in quanto il coseno di 90 gradi corrisponde a 0. \\
Ovviamente nel caso di di vettori espressi in modulo e argomento è inutile tutto questo discorso in quanto per calcolare l'angolo compreso basterà fare la differenza tra gli angoli dei due vettori, ma nel caso in cui invece fossimo nella rappresentazione per coordinate tutto può diventare estremamente utile, e in quel caso la formula del prodotto scalare (nota : ciò vale solo nei reali ) corrisponde a questa :
$$\overrightarrow{U} * \overrightarrow{V} = |\overrightarrow{U}||\overrightarrow{V}| * \cos(\widehat{UV})$$
Da cui è possibile ricavare l'angolo in questo modo :
$$\widehat{UV} = \arccos \left( \frac{ |\overrightarrow{U}||\overrightarrow{V}|}{\overrightarrow{U} * \overrightarrow{V} } \right)$$ \\\
\subsection{Cinematica}
\hspace*{1cm}
\subsubsection{Traiettoria}
La \textbf{traiettoria} è una funzione nello spazio \underline{f(x,y,z) = 0} che indica tutti i punti dello spazio che sono stati, sono e saranno percorsi da un corpo durante il suo movimento. Ho usato il passato, il presente e il futuro contemporaneamente perché noi non sappiamo il tempo necessario a raggiungere un punto o ad effettuare uno spostamento, sempre se il corpo si muovo solo in un verso e che non faccia avanti e indietro, cosa che può accadere, ma sappiamo solo che il corpo durante il suo movimento passerà dal quel punto.
\subsubsection{Equazione oraria}
Se invece desiderassimo mettere in relazione la posizione di un corpo coni il tempo allora dovranno sarà necessaria un'equazione oraria, ovvero appunto un equazione che mette il relazione il punto dove si trova il corpo con il tempo e si indica nel seguente modo (per i tre assi cartesiani ) :
\begin{itemize}
\item $x(t) = ....$
\item $y(t) = ....$
\item $z(t) = ....$
\end{itemize}
Dove nel termine a destra appare appunto la posizione su uno degli assi in funzione del tempo, mentre a sinistra apparirà un normale membro di un'equazione. \\
Detto questo si può indicare come le coordinate di un punto in funzione del tempo siano : 
$$\overrightarrow{P} = (x(t),y(t),z(t))$$ 
\subsubsection{Velocità e accelerazione}
\hspace*{1cm} \\
\textbf{\textit{Velocità}} \\ \\ 
Data un'equazione oraria, viene definita come \textbf{velocità} la derivata dell'equazione oraria in funzione del tempo.
Molto importante osservare che la velocità non è un semplice scalare ma bensì un vettore formato da un modulo, che indica la "quantità di velocità" ed una direzione, che indica la direzione verso la quale si sposterà il corpo. Vedendo la velocità come un vettore, è importante constatare che le componenti che compongono il vettore sono appunto gli spostamenti che tale velocità produce lungo gli assi cartesiani (quindi avrà una componente x, una y e una z). \\ \\
\textbf{Importante: }La velocità è un vettore che è sempre parallelo alla traiettoria del corpo. \\ \\
\textbf{\textit{Velocità Istantanea}} \\ \\ 
La \textbf{velocità istantanea} di un corpo viene definito come 
$$\lim_{t \to 0} \frac{P_2(x_2(t),y_2(t),z_2(t)) - P_1(x_1(t),y_1(t),z_1(t))}{\Delta \; t}$$
Con $P_1 \; e \; P_2$ due punti lungo la traiettoria del corpo. \\ 
La velocità istantanea è per definizione sempre (in qualunque punto) \textbf{tangente} alla traiettoria nel punto nel quale vogliamo calcolare la velocità istantanea. \newpage
\textbf{\textit{Accelerazione}} \\ \\ 
L'accelerazione viene definita come la \textbf{variazione di velocità} nel tempo ed è quindi la derivata prima della velocità del tempo, e di conseguenza la derivata seconda dell'equazione oraria, ad esempio :
$$\overrightarrow{a} = \frac{d\overrightarrow{V}}{dx} = \frac{d^2 \overrightarrow{P}}{dx}$$
Ovviamente anche l'accelerazione è un vettore che avrà come componenti : 
\begin{itemize}
\item Come  x avrà la derivata della componente x del vettore velocità
\item Come  y avrà la derivata della componente y del vettore velocità
\item Come  z avrà la derivata della componente z del vettore velocità
\end{itemize}
Per visualizzare questo meglio 
$$\overrightarrow{a} = (\frac{dV_x(t)}{dt},\frac{dV_y(t)}{dt},\frac{dV_z(t)}{dt})$$
\subsubsection{Esempi}
Supponiamo di avere l'equazione oraria $x(t) = A\cos(\omega t)$, trovare $\overrightarrow{V}$ e $\overrightarrow{a}$ : \\ 
Calcoliamo la velocità effettuando la derivata dell'equazione oraria :
$$V(t) = \frac{d}{dx}x(t)= \frac{d}{dx} A\cos(\omega t) = -\omega A \sin(\omega t)$$
E dalla velocità calcoliamo l'accelerazione(sempre in funzione del tempo):
$$a(t) = \frac{d}{dx}V(t) = \frac{d}{dx} -\omega A \sin(\omega t) = -\omega^2 A \cos (\omega t)$$
Notando il risultato salta subito all'occhio che l'accelerazione del corpo dipende dalla posizione del corpo stesso infatti :
$$a(t) = -\omega^2 A\cos (\omega t) =  -w^2x(t)$$
Quando abbiamo che l'accelerazione di un corpo dipende dalla sua posizione si dice che abbiamo un \textbf{moto armonico}. \\
Più in generale se $\frac{d^2 f(t)}{dt^2} = -Kf(t) $ si ha un moto periodico (f(t) non importa che sia obbligatoriamente x(t), ma può essere una qualunque funzione nel tempo). \newpage
\section{Appunti 05 marzo 2015}
\subsection{Osservazione}
Come si può osservare da questa equazione oraria
$$x(t) = 2t^2 + 3t^3$$
Applicando l'operazione di derivata viene ridotto il grado dell'equazione di un'unità per derivazione ( pura e semplice matematica ) che ci permettono di fare alcune considerazioni :
\begin{enumerate}
\item Se un'equazione oraria è di grado 3 a(t) lineare
\item Se un'equazione oraria è di grado 2 a(t) costante e v(t) lineare (uniformemente accelerata lungo quell'asse)
\item Se un'equazione oraria è di grado 1 a(t) = 0 e v(t) costante (ovviamente lungo l'asse dell'equazione)
\end{enumerate}
Infatti tornando all'equazione di prima notiamo che $v(t) = 4t + 9t^2$ e $a(t) = 4 + 18t$ (che è  lineare).
\subsection{Esercizi}
\subsubsection{Primo esercizio}
Prendiamo un sistema di due equazioni orarie (che descrivono quindi lo spostamento di un corpo lungo due assi)
\[
\left\{ \begin{array}{lr}
	x(t) = 2t^2 -3t^3 \\ 
	y(t) = 5t + 4 \\
\end{array} \right.
\]
trovare l'angolo tra vettore velocità e vettore accelerazione all'istante 1. \\ 
Innanzi tutto procediamo a trovare i vettori velocità e accelerazione per entrambi gli assi :
\[
\left\{ \begin{array}{ll}
	V_x(t) = 4t -9t^2  \qquad \Longrightarrow a_x(t) = 4 - 18t\\ 
	V_y(t) = 5  \qquad \quad \qquad \Longrightarrow a_y(t) = 0
\end{array} \right.
\]
Adesso non ci rimane che sostituire 1 a t per ottenere il vettore velocità e accelerazione :
\[
\left\{  \begin{array}{ll}
	V_x(1) = 4(1) -9(1)^2 = 4 - 9 = -5  \quad \Longrightarrow a_x(1) = 4 - 18(1) = 4- 18 = -14\\ 
	V_y(1) = 5  \qquad \qquad \quad \qquad \qquad \quad \qquad \Longrightarrow a_y(1) = 0
\end{array} \right.
\]
Ecco che abbiamo ottenuto che V(1) = (1,1) e a(1) = (-14,0). \\  
Per finire non ci rimane che ricordarsi che il prodotto scalare contiene il coseno dell'angolo compreso e applicare la formula per trovare un angolo a partire dal prodotto scalare :
$$\widehat{V(1)a(1)} = \arccos \left(\frac{V(1) a(1)}{|V(1)||a(1)|} \right) =\arccos \left(\frac{70}{14\sqrt{50}} \right) = \arccos \left( \frac{1}{\sqrt{2}} \right) = 45 ^ \circ$$ 
\subsubsection{Secondo esercizio}
Calcolare la traiettoria dato il seguente sistema di equazioni orarie :
\[
\left \{ \begin{array}{lr}
	x(t) = 6t^2 + 3  \Longrightarrow x = 6\frac{y}{2} + 3 \Longrightarrow 2x- 3y^2 - 4 = 0\\
	y(t) = 2t  \; \qquad \Longrightarrow t = \frac{y}{2}
\end{array} \right.
\]
Dalla seconda equazione ricavo t (che poi andrà sostituito per farlo sparire, perché un equazione oraria non presenta il tempo) e lo sostituisco, ricavando l'equazione oraria che poi verrà sistemata con semplici passaggi algebrici(equazione oraria ottenuta : $ 2x- 3y^2 - 4 = 0$ ).
\subsubsection{Terzo esercizio}
Dato un sistema di equazioni orarie lungo due assi trovare, velocità,accelerazione e traiettoria:
\[
\left \{ \begin{array}{lr}
	x(t) = A\cos(\omega t) \Longrightarrow V_x(t) = -\omega A \sin(\omega t) \Longrightarrow 
				a_x(t) = -\omega^2 A^2\cos(\omega t)\\
	y(t) = A\sin(\omega t) \, \Longrightarrow V_y(t) = \omega A\cos(\omega t) \; \;\,\Longrightarrow a_y(t) = -\omega^2 A^2 \sin(\omega t)
\end{array} \right.
\]
A questo punto abbiamo trovato velocità e accelerazione, non ci rimane che trovare la traiettoria, ma questa volta non è possibile sostituire come prima, ma visto che è presente il seno e il coseno possiamo elevare tutto al quadrato e usare le proprietà di seno e coseno.
\[
\left \{\begin{array}{lr}
	x^2(t) = A^2\cos^2(\omega t) \Rightarrow x^2(t) = A^2(1 - \sin^2(\omega t)) \Rightarrow x^2 = A^2 (1 -\frac{y^2}{A^2})\Rightarrow x^2 + y^2 = A^2\\
	y(^2t) = A^2\sin^2(\omega t)  \Rightarrow \sin^2(\omega t) = \frac{y^2}{A^2}
\end{array} \right.
\]
Dall'equazione oraria($x^2 + y^2 = A^2$) si vede che il corpo si muove secondo una traiettoria circolare (moto circolare) di raggio A. \\
Essendo questo un moto curvilineo siamo sicuri che la velocità non sia costante, perché anche se della velocità ne è costante il modulo, non lo sarà sicuramente la direzione (che è tangente alla traiettoria),calcoliamo quindi il modulo della velocità :
$$|\overrightarrow{V}| = \sqrt{V_x ^2 + V_y ^ 2} = \sqrt{\omega^2 A^2 \sin^2(\omega t) + \omega^2 A^2 \cos^2(\omega t)} = \omega A$$
Dato che A è il raggio si può constatare che il  modulo della velocità del corpo dipende dal raggio della sua traiettoria. \\
Adesso calcoliamo il modulo dell'accelerazione:
$$|\overrightarrow{a}| = \sqrt{\omega^4 A^2\cos^2(\omega t) + \omega^4 A^2 \sin^2(\omega t)} = \omega^2 A $$
E dato che A corrisponde al raggio e dato che il la direzione dell'accelerazione sia opposta al raggio(Si vede dal segno sulle equazioni che legano l'accelerazione lungo gli assi) si può dire in definitiva che $\overleftarrow{a} = -\omega^2R$. \\
Quando abbiamo una accelerazione che di direzione opposta rispetto al raggio (quindi quando "tira" l'oggetto verso l'interno) si dice che abbiamo un'\textbf{accelerazione centripeta}. \\
\subsubsection{Conclusione}
Dato che per definizione la velocità è sempre tangente alla traiettoria, e dato che in questo caso siamo in una circonferenza si può notare che la velocità sia perpendicolare al raggio (e quindi all'accelerazione).Per dimostrarlo basta fare il prodotto scalare tra il vettore che rappresenta il raggio $\overleftarrow{R} = (A\cos(\omega t),A\sin(\omega t)$ e quello velocità : 
$$\overrightarrow{R} * \overrightarrow{V} = -\omega A^2 \cos(\omega t) \sin(\omega t) +\omega A^2 \cos(\omega t) \sin(\omega t) = 0 $$
e quindi proviamo anche che il vettore velocità è perpendicolare al vettore accelerazione:
$$\overrightarrow{a} * \overrightarrow{V} = \omega^3A^2\cos(\omega t) \sin(\omega t) -\omega^3A^2\cos(\omega t) \sin(\omega t) = 0$$
Per concludere notiamo che se invece il moto non è circolare ma ellittico allora il vettore accelerazione non sarà perpendicolare al vettore velocità, ma sarà sempre parallelo al vettore raggio dell'ellisse.
\section{Appunti del 06 marzo 2015 (esercitazione)}
\subsection{Prodotto scalare}
\subsubsection{Breve Nota}
Il prodotto scalare è massimo quando due vettori sono paralleli (perché l'angolo compreso è zero e il coseno di 0 è 1) ed è nullo quando sono ortogonali (coseno di $90^\circ$ = 0).
\subsection{Prodotto vettoriale}
\subsubsection{Formula e dimostrazione}
L'operazione di prodotto vettoriale tra due vettori restituisce un terzo vettore e si rappresenta nel seguente modo:
$$\overrightarrow{c} = \overrightarrow{a} \wedge \overrightarrow{b}$$
$$|\overrightarrow{c}|=|\overrightarrow{a}||\overrightarrow{b}|\sin(\widehat{ab})$$
Quest'ultima formula si ottiene dalla seguente dimostrazione:$(a \wedge b)^2 = ((a \wedge b)*(a \wedge b) = a^2b^2 - (ab)^2 = a^2b^2 -(|a||b|\cos(\widehat{ab}))^2 = a^2b^2(1-\cos^2\widehat{ab})) = a^2b^2 \sin^2(\widehat{ab})$ e quindi si conclude dicendo che se $(a \wedge b)^2 = a^2b^2 \sin^2(\widehat{ab}) $ allora $\sqrt{(a \wedge b)^2} = \sqrt{a^2b^2 \sin^2(\widehat{ab})} $ da cui si ottiene che : $|(a \wedge b)| = |a||b|\sin(\widehat{ab})$. 
\subsubsection{Calcolo del prodotto vettoriale}
Per calcolare il prodotto vettoriale si usa una matrice avente per righe i vettori che servono per fare l'operazione, mettendo invece per colonne i versori i,j,k.
\subsubsection{Identità vettoriali}
\begin{enumerate}
\item $(\overrightarrow{a} \wedge \overrightarrow{b})\wedge \overrightarrow{c} = (\overrightarrow{a}*\overrightarrow{b})*\overleftarrow{b} - (\overrightarrow{b}*\overrightarrow{c})*\overrightarrow{a} $
\item $(\overrightarrow{a} \wedge \overrightarrow{b})*(\overrightarrow{c} \wedge \overrightarrow{d}) = (\overrightarrow{a}*\overrightarrow{c})*(\overrightarrow{b}*\overrightarrow{d})-(\overrightarrow{a}*\overrightarrow{d})*(\overrightarrow{b}*\overrightarrow{c})$
\item $(\overrightarrow{a} \wedge \overrightarrow{b}) \wedge(\overrightarrow{c} \wedge \overrightarrow{d})= $
\item $(\overrightarrow{a} \wedge \overrightarrow{b})*\overrightarrow{c}$ restituisce zero se i vettori sono complanari,perché il prodotto vettore restituisce un vettore ortogonale al piano ma se anche il terzo vettore appartiene allo stesso piano allora il prodotto scalare per il vettore ortogonale a tale piano sarà uguale a 0 ,altrimenti restituisce un valore corrispondente al volume del parallelepipedo costruito sui tre vettori.
\end{enumerate}
\subsection{Cambio sistema di riferimento}
\subsubsection{Definizione}
Per cambio di sistema di riferimento si intende il passaggio da un sistema di basi sul quale è costruito un vettore ad un altro, calcolandone dunque il corrispettivo vettore associato al nuovo sistema.\\
Per calcolare il corrispettivo vettore sul nuovo sistema di riferimento basterà scrivere il vettore come combinazione lineare dei nuovi i,j,k. \\ \\
\section{Appunti 10 marzo 2015}
\subsection{Equazioni differenziali}
Quando si parla di equazioni differenziali si intendono quella tipologia di equazioni dove compaiono come termini anche le derivate di funzioni. In fisica le equazioni differenziali sono molto ricorrenti, perché basta pensare alla velocità come derivata dello spazio nel tempo e l'accelerazione come la derivata nella velocità nel tempo ed ecco che se un moto ha una velocità che dipende punto in cui si trova abbiamo già un equazione differenziale perché
$$\frac{dx(t)}{dt} = Kx(t)$$
Oppure se è l'accelerazione a dipendere dallo spazio o viceversa, oppure ancora se lo spazio percorso dipende sia dalla velocità che dall'accelerazione (in questo caso abbiamo una differenziale del secondo ordine) come vedremo  nelle equazioni caratteristiche dei moti.
\subsection{Equazioni dei moti}
\subsubsection{Moto uniformemente accelerato}
L'equazione oraria di un moto uniformemente accelerato è un'equazione differenziale del secondo ordine (perché l'accelerazione è la derivata seconda dello spazio nel tempo) che viene scritta in questo modo
$$x(t) = x_0 + v_0t + \frac{1}{2}at^2$$
dove $x_0$ è il punto di partenza del corpo (ovvero dove si trova al tempo 0), $v_0$ è la velocità del corpo al tempo 0 e a è il valore di accelerazione del corpo. Infatti se calcoliamo le derivate successive troviamo che 
$$v(t) = v_0 + at $$ $$ a(t) = a$$ 
che dimostra appunto le affermazioni precedenti (infatti a(t) = a indica accelerazione costante).
\subsubsection{Moto rettilineo uniforme}
Un moto rettilineo uniforme è un moto nella quale ad essere costante è la velocità (che ricordiamo è un vettore, e che quindi impone anche una traiettoria rettilinea per non cambiare direzione, che indicherebbe presenza di accelerazione) e si presenta con questa equazione oraria:
$$x(t) = x_0 + v_0t$$
Anche in questo caso sia $x_0$ che $v_0$ hanno lo stesso valore di prima, l'unica differenza è che non compare l'accelerazione, infatti se deriviamo otteniamo questo:
$$v(t) = v_0$$
$$a(t) = 0$$
Ovvero proprio ciò che volevamo, cioè velocità costante e accelerazione pari a 0.
\subsubsection{Esempi}
\newpage
\section{Appunti 11 marzo 2015}
\subsection{Moto relativo}
\subsubsection{Definizione}
Si parla di moto relativo quando vogliamo studiare il movimento di un corpo che si muove all'interno di un sistema che è anch'esso in movimento (esempio : studiare il comportamento di un corpo all'interno di un treno in movimento). \\
Innanzitutto è opportuno indicare sia le formule per calcolare la velocità e l'accelerazione assolute di un corpo (ovvero la velocità e l'accelerazione"totali" di un corpo all'interno di un sistema in movimento).
$$V_a = V_{trasc}  + V_{rel}$$
$$a_a = a_{trasc}  + a_{rel}$$
Dove con il pedice a si intende assoluta, con il pedice "trasc" si intende di trascinamento (ovvero quella del sistema, ad esempio quella del treno) e con il pedice "rel" si intende quella relativa al solo corpo all'interno del sistema. \\
Dalle formule si vede che l'accelerazione(e la velocità) assoluta è la somma di due componenti, e quindi di conseguenza l'accelerazione/velocità relativa è la differenza tra quella assoluta e quella di trascinamento. Proprio questa differenza introduce il concetto di accelerazione "fittizia", ovvero un'accelerazione che sembra esistere ma che in realtà e solo frutto della differenza di altre due (come ad esempio l'accelerazione centrifuga, che in realtà è solo fittizia). \\
Questa formula spiega anche come mai se cadiamo a terra da un ascensore, anch'esso in caduta, per noi che siamo al suo interno sembra che non ci sia gravità, in quanto l'accelerazione assoluta è "g", ma anche l'accelerazione relativa è "g" perché tutto l'ascensore sta cadendo al suolo, quindi g-g = 0, dimostrando che noi in quel momento non siamo sottoposti ad alcuna accelerazione, in maniera apparente. \\
Tornando sui sistemi di rotazione (ovvero quei sistemi che si muovono roteando), l'accelerazione centrifuga è un'altra accelerazione apparente, in quanto non è altro che l'opposto dell'accelerazione di trascinamento (che in questo caso l'accelerazione centripeta).
Anche l'accelerazione di coriolis è apparente, in quanto un corpo all'interno di un sistema di rotazione (come anche la terra) sembra che si muova in direzione opposta al senso di rotazione, ma in realtà il corpo sotto che "scivola" via dal corpo, che non si muovo in modo congiunto con il sistema.
\subsection{Forze}
Quando parliamo di forza viene subito da pensare alla formula 
$$F = ma$$
che però nel caso generale risulta \textbf{errata}, in quanto in realtà la forza è la derivata della quantità di moto nel tempo, quindi realmente 
$$\overleftarrow{F} = \frac{d \overrightarrow{P}}{dt}$$
la quantità di moto vene definita come il prodotto tra la massa e  la velocità
$$\overrightarrow{P} = m\overrightarrow{V}$$
da cui si ricava 
$$\overrightarrow{F} = \frac{d\overrightarrow{V}}{dt}m + V\frac{d}{dt}m = \overrightarrow{a}m + V\frac{d}{dt}m$$
Quindi in definitiva la forza è il prodotto tra la massa e l'accelerazione se e solo se la derivata della massa nel tempo è uguale a zero, ovvero se il corpo ha massa costante nel tempo, altrimenti la forza viene definita come una variazione della quantità di moto nel tempo.
\subsubsection{Impulso}
Viene definito impulso invece la dipendenza funzionale della forza dal tempo e si esprime con la seguente formula
$$\overrightarrow{I} = \int_{t_i}^{t_f} \sum \overrightarrow{F}$$
Da questa definizione si vede che l'impulso è un vettore il cui modulo è l'area sottesa alla curva forza-tempo in un intervallo $\Delta t = t_f - t_i$.
\subsection{Conservazione quantità di moto}
Ci sono alcuni casi in cui la quantità di moto si conserva, e questi casi sono :
\begin{itemize}
\item Se le forze esterne sono nulle (la variazione di tempo può essere nulla, non nulla o infinitesima)
\item Se la variazione di tempo è nulla
\item Se la forza non è nulla e la variazione di tempo non è nulla
\end{itemize}
In quest'ultimo caso è necessario controllare il tipo di forze esterne, se sono \textbf{impulsive} allora la quantità di moto varia, se non lo sono allora la quantità di moto rimane costante.
\section{Appunti 12 marzo 2015}
\subsection{Caduta dei gravi}
\textbf{Nota:} Quando noi studiamo il comportamento di un corpo escludiamo la forza di attrazione gravitazionale del sole, non perché quest'ultima sia talmente piccola da essere trascurabile ma perché noi stessi siamo in "caduta libera" rispetto al sole, ovvero "tutta" la forza di attrazione del sole viene usata per attrarre l'intero pianeta, quindi sempre per il moto relativo a noi risulta una forza nulla (come nell'esempio dell'ascensore in caduta e le persone al suo interno che fluttuano come se non ci fosse gravità). \\ \\
Quando si studia la caduta dei gravi si deve tenere conto non solo di uno spostamento lungo l'asse verticale (ovvero la caduta verticale di un corpo ) ma anche lo spostamento che esso può avere lungo l'asse delle x, dovuto alla forza di Coriolis(che ricordo essere una forza fittizia dovuta alla rotazione, in questo caso, della Terra), che nella maggioranza dei casi spingerà il corpo verso est.
\subsubsection{Esempio}
Esempio di una caduta di corpo con spostamento dovuto alla forza di Coriolis: \\ 
In questo esempio l'accelerazione lungo l'asse z è uguale a -g (ovvero cade ad un accelerazione costante pari a g) con un equazione oraria di questo tipo
\[ \begin{array}{lr}
z(t) = -\frac{1}{2}gt^2 + h \\
v_z(t) = -gt \\
a_z(t) = -g \\
\end{array}
\]
Dal prodotto vettore tra $V_z$ e $-2\omega$ si vede che l'oggetto in caduta si muove lungo l'asse x in direzione est e il suo spostamento sarà pari al modulo che è 
$$-2\overrightarrow{\omega}*\overrightarrow{V_z} = |\omega||V_z|\sin(\frac{\pi}{2} + \lambda) = |\omega||V_z|\cos(\lambda)$$
quindi il vettore generato dal prodotto vettore avrà queste componenti : $(0,2|\omega||V_z|\cos(\lambda),V_z)$ che corrisponderanno alle velocità lungo i tre assi.
Adesso non ci rimane che ricavare l'equazione oraria lungo l'asse delle y che è uguale a 
$$y(t) = \frac{1}{3}gt^3\cos(\lambda)$$
\section{Appunti 13 marzo 2015 (esercitazione)}
\subsection{Rappresentazione vettore}
\subsubsection{Coordinate cartesiane (con versori)}
Un vettore può essere rappresentato mediante coordinate cartesiane con l'ausilio di versori, che ci consento di vedere il nostro vettore come somma di altri n versori, dove n è uguale alla dimensione in cui stiamo studiando il problema. Quindi preso un vettore a caso $V = (A,B)$ la sua rappresentazione per mezzo di versori è 
$$V = Au_x + Bu_y$$
Dove $u_x$ e $u_y$ sono rispettivamente versori lungo l'asse x e y.
\subsubsection{Coordinate polari}
\subsection{Velocità}
\subsubsection{Coordinate cartesiane (con versori)}
\subsubsection{Coordinate Polari}
\section{Appunti 17 marzo 2015}
\subsection{Contatto tra corpi}
Nella vita di tutti i giorni, capita di continuo di vedere due oggetti che si toccano, ma in realtà questa è solo una nostra "illusione" dovuta dalla bassa risoluzione dei nostri occhi (ovvero dalla minima dimensione di oggetto visibile dal nostro occhio) in quanto se andassi a magnificare la zona di contatto fino a livello atomico è possibile notare che in realtà gli atomi dei corpi si respingono, evitando quindi ogni qual forma di contatto.
\subsection{Forza di attrito}
La forza di attrito tra due corpi è dovuta dal fatto che le superfici dei due corpi non sono perfettamente lisce (ovviamente si parla a livello atomico ) ma forma una specie di percorso seghettato. Questo vuol dire che se due corpi si toccano la pressione che fa il corpo che si appoggia sulla superficie non viene distribuita lungo tutta la superficie di contatto, ma invece essa è concentrata lungo solo alcuni punti di contatto ( quelle che possono essere viste come le punte del percorso seghettato ), che subendo una cosi elevata pressione si "fondono" con la superficie di appoggio, ovvero penetrano nella superficie. La forza di attrito non è altro che la forza che oppone questa specie di "micro-saldatura" al movimento. \\ 
\subsubsection{Attrito statico}
La forza di attrito statica è una \textbf{forza di reazione}, cioè una forza che entra in gioco solo nel momento in cui si applica una forza esterna al corpo, trovando appunto la resistenza posta dalla forza di attrito.Questo vuol dire finché non applico forze che fanno muovere il corpo, la forza di attrito sarà pari a 0. Ciò infatti si capisce chiaramente se consideriamo $F = \frac{dP}{dt}$, infatti dato che dP = 0, perché non si muove, allora tutte le forze in gioco ( su quell'asse) devono essere pari a 0, e quindi anche la forza di attrito.\\
La forza di attrito statica è anche una forza non costante, ovvero che non ha sempre il solito valore ( a parità di coefficiente e di forza normale) in quanto se cosi fosse, dato che è una forza di reazione, nel momento in cui applichiamo al corpo una forza minore di quella di attrito, esso dovrebbe muoversi nella direzione opposta, ma ciò non avviene perché la forza di attrito assume un valore pari alla forza che noi esercitiamo, ovviamente fino ad un certo limite, soltanto di direzione opposta. \\
La forza massima che può essere generata dall'attrito statico è pari a 
$$F_{as} =\mu_s |N| $$
dove N è la forza normale (ovvero quella ortogonale al piano, nel caso di un piano orizzontale la forza peso) e $\mu_s$ invece il coefficiente di attrito statico.
\subsubsection{Attrito dinamico}
La forza di attrito dinamico è una forza di attrito costante dal valore minore della forza di attrito statico, in quanto muovendosi le superficie a contatto dei corpi hanno meno tempo per effettuare quelle "micro-saldature" che quindi saranno meno resistenti. La formula per l'attrito dinamico è invece 
$$F_{ad} = \mu_d|N|$$
dove $\mu_d$ è il coefficiente di attrito dinamico ($\mu_d < \mu_s < 1$).
\subsubsection{Attrito volvente}
L'attrito volvente invece è un tipo di attrito che si incontra negli oggetti sferici ( o a forma di ruota) ed è molto minore anche dell'attrito dinamico, in quanto durante la rotazione del corpo viene spostato molto velocemente il punto di contatto, riducendo quindi l'efficacia del contatto tra le due superfici, e quindi l'attrito.
\subsection{Lavoro}
Viene definito come la variazione del lavoro il prodotto scalare tra la forza e lo spostamento 
$$dL = F * ds$$
mentre viene definito \textbf{Lavoro} l'integrale in linea (ovvero fatto punto per punto lungo la traiettoria) del prodotto scalare tra la forza e lo spostamento
$$L = \int_{linea}F * ds$$
Nel caso in cui volessimo calcolare il lavoro lungo la traiettoria tra a e b allora otteniamo
$$L_{a \to b} = \int_{a_{linea}}^b F *ds$$
\subsection{Energia cinetica}
Il teorema dell'energia cinetica dice che nel caso in cui la massa sia costante (quindi $F = ma$) allora la variazione del lavoro può essere espressa nel seguente modo:
$$dL = m\frac{dV}{dt}*ds = m*dV*\frac{ds}{dt} = m*dV*V$$
posta invece l'energia cinetica come variabile k il suo valore sarà 
$$k = \frac{1}{2}mV^2$$
e la sua variazione
$$dk= \frac{m}{2}d(\overrightarrow{V} * \overrightarrow{V}) = \frac{m}{2}d\overrightarrow{V}\overrightarrow{V} * \frac{m}{2}\overrightarrow{V}d\overrightarrow{V}$$
Si può anche esprimere il lavoro come la variazione dell'energia cinetica $L = \Delta k$
\subsubsection{Forze conservative}
Una forza si dice conservativa se il lavoro dipende solo dalla posizione iniziale e finale del corpo, e non dalla traiettoria che esso ha fatto per raggiungere tale posizione dalla partenze. \\ 
Un esempio di forza conservativa è la forza peso, la forza elastica oppure la forza di attrazione gravitazionale, in quanto il movimento lungo la circonferenza non richiede lavoro, solamente quello lungo il raggio. \\
Si può quindi dire in definitiva che 
$$\Delta U + \Delta R = 0$$
$$U + R = costante$$
con U che corrisponde all'energia potenziale.
\subsubsection{Conservazione energia meccanica}
In un sistema reale la conservazione dell'energia meccanica avviene se tutte le forze sono conservative oppure se quelle che non lo sono hanno valore 0 (ovvero non influiscono). \\
Schematizzando quindi i passi da fare per capire se un sistema conserva energia meccanica sono i seguenti : 
\begin{enumerate}
\item Guardare tutte le forze in gioco
\item Controllare se tutte sono conservative (se si abbiamo finito)
\item Se non lo sono verificare che le forze non conservative abbiano valore > 0 
\item Se la risposta è si allora il sistema non conserva energia, altrimenti la conserva.
\section{Appunti 18 marzo 2015}
\subsection{Forze conservative}

\subsection{Esercizio 1}
\subsubsection{Testo}
\subsubsection{Risoluzione}
\subsection{Esercizio 2}
\subsubsection{Testo}
\subsubsection{Risoluzione}
\subsection{Pendolo}
\section{Appunti 19 marzo 2015 (esercitazione)}
\subsection{Esercizio 1}
\subsubsection{Testo}
\subsubsection{Risoluzione}
\subsection{Esercizio 2}
\subsubsection{Testo}
\subsubsection{Risoluzione}
\end{enumerate}
\end{document}
