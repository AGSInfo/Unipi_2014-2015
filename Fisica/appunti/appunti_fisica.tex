\documentclass[fontsize = 20px, paper = a4]{article}
\author{Alessandro Sieni, Gianluca Mondini}
\title{Appunti Fisica Generale}
\date{\today}
\usepackage{amsfonts}
\usepackage[utf8]{inputenc}
\begin{document}
\maketitle
\newpage
\tableofcontents
\newpage
\section{Appunti 03/03/2015}
\subsection{Ordini di grandezza}
Quando si parla di ordini di grandezza intendiamo non una precisa quantità ma un'indicazione utile ad effettuare delle stime che non devono essere necessariamente precise.
\subsection{Calcolo dimensionale}
\subsubsection{Definizione}
Il calcolo dimensionale ci permette di lavorare solo con le dimensioni che compongono le componenti da studiare e ci permette di verificare se alcuni procedimenti o calcoli sono corretti dal punto di vista dimensionale, ovvero se la dimensione ottenuta è coerente con quella di ciò che dobbiamo calcolare.
Nel caso di alcuni casi semplici è possibile ottenere delle formule senza alcun procedimento, ma solo mediante l'utilizzo del caso dimensionale.
\subsubsection{Esempi}
Esaminiamo in questo esempio la formula per ottenere il tempo di caduta di un corpo da una fissata altezza.
Solo studiando in maniera puramente intuitiva il fenomeno una persona può ipotizzare che la velocità possa dipendere dall'accelerazione di gravita( g ) dal tempo di caduta (t) e dall'altezza dalla quale il corpo viene lanciato, quindi riportando queste tre grandezze sotto forma di dimensioni otteniamo che:
$$g = [V][T]^-1 = [L][T]^-2$$
e dato che $[V] = [L][T]^-1$ dobbiamo riottenere la stessa dimensione, e per l farlo dobbiamo trovare gli esponenti $x,y,z$ da assegnare a g,t,h che si andranno a moltiplicare, che ci permettano di ritrovare la stessa dimensione, in poche parole :
$$[L][T]^-1 = \underbrace{([L][T]^-2)^x}_g \underbrace{[T]^y}_t \underbrace{[L]^z}_h$$
Da questo procediamo in modo algebrico ottenendo
$$[L][T]^-1 = \underbrace{[L]^x[T]^{-2x}}_g \underbrace{[T]^y}_t \underbrace{[L]^z}_h$$
da cui abbiamo un sistema a due equazioni ma a tre variabili (x,y,z) in quanto
$$
\begin{array}{lr}
   x+z = 1 \\
   y - 2x = -1
  \end{array}
$$
Per risolvere questo problema proviamo a non considerare l'altezza come fattore determinante per calcolare la velocità e troviamo quindi come si comporta il tempo di caduta rispetto a g e h.
$$[T] = [T]^y ([L][T]^-2)^x$$
ottenendo stavolta un sistema a due equazioni e due incognite che ha come soluzione $x = -\frac{1}{2}$ e $y = \frac{1}{2}$. Infatti la formula del tempo di caduta di un corpo è proprio $\sqrt{\frac{h}{g}}$-. \\
\subsection{Vettori}
\hspace*{0.1cm}
\subsubsection{Introduzione}
\hspace*{0.1cm} \\ 
In fisica l'utilizzo di vettori invece che di semplici scalari è fondamentale in quanto ci permettono di rappresentare in modo corretto la realtà (ad esempio non è possibile rappresentare la posizione di un corpo con un solo numero, ma ne sono necessari 3, uno per ogni asse cartesiano x,y,z).
\subsubsection{Operazioni sui vettori}
Le operazioni che è possibile effettuare sui vettori sono le seguenti :
\begin{itemize}
\item Somma
\item Prodotto tra un vettore ed uno scalare
\item Prodotto scalare tra due vettori
\item Prodotto vettoriale
\end{itemize}
\textbf{\textit{Somma}} \\  \\
La somma tra due vettori restituisce un vettore le cui componenti corrispondono alla sommatoria delle componenti relativi alla solita posizione, esempio :
$$(1,2,3) + (7,-3,4) = (1+ 7,2 + (-3),3+4) = (8,-1.7)$$ \\  
\textbf{\textit{Prodotto di un vettore per uno scalare}} \\ \\
Il prodotto di un vettore per uno scalare si ottiene moltiplicando ciascuna componente per lo scalare, esempio :
$$3 * (4,2,5) = (4 * 3, 2 * 3, 5* 3) = (12,6,15)$$ \\ 
\textbf{\textit{Prodotto scalare tra due vettori}} \\ \\
Il risultato del prodotto scalare tra due vettori sarà appunto uno scalare che corrisponderà alla sommatoria del prodotto delle relative componenti dei due vettori, esempio presi $U = (1,2,3)$ e $V=(3,4,5)$ il prodotto scalare tra U e V sarà :
$$U*V = \sum_{i = 1}^3 (U_i * V_I) = (1* 3) + (2*4) + (5 * 3) = 26$$ \\ 
\textbf{\textit{Prodotto vettoriale tra due vettori}} \\ \\
\subsubsection{Rappresentazione di un vettore}
\hspace*{5cm} \\
Per rappresentare un vettore ci sono 4 possibili modi : 
\begin{itemize}
\item Mediante le coordinate
\item Mediante una freccia nel piano
\item Mediante il modulo del vettore e il suo argomento
\item Mediante l'utilizzo di versori
\end{itemize}
\section{Appunti 04/03/2015}
\subsection{Rappresentazione di un vettore}
\hspace*{5cm} \\
Per rappresentare un vettore ci sono 4 possibili modi : 
\begin{itemize}
\item Mediante le coordinate
\item Mediante una freccia nel piano
\item Mediante il modulo del vettore e il suo argomento
\item Mediante l'utilizzo di versori
\end{itemize}
\subsubsection{Coordinate}
Quando si rappresenta un vettore mediante coordinate si esplicita una tupla di n numeri l'esempio più classico è $V = (1,2,3)$ \newpage
\subsubsection{Freccia nel piano}
Quando si rappresenta un vettore con una freccia si utilizzano le stesse coordinate illustrate sopra ma le si proiettano in un ipotetico spazio n-dimensionale (ovviamente questo nei modelli matematici, perché in quelli fisici non si supera la terza dimensione, eccetto alcuni rari casi) e si traccia una freccia che parte dall'origine degli assi e che arriva proprio nel punto dello spazio avente come coordinate cartesiane le componenti del vettore. \\ 
\subsubsection{Modulo e Argomento}
Quando si rappresenta un vettore con la notazione modulo e argomento intendiamo esprimere il modulo del vettore (nel caso sia rappresentabile in uno spazio con una freccia la sua lunghezza partendo dall'origine degli assi) e dato che la sola lunghezza non è sufficiente (in quanto dando solo la lunghezza di una freccia si potrebbe intendere una circonferenza di centro 0 e raggio uguale al modulo della freccia) viene espresso anche l'angolo che forma il vettore con un asse (nel caso sia su un piano) e con 2 assi (nel caso sia su uno spazio tridimensionale), esempio:
$$\overrightarrow{U} = (3,30^\circ)$$
In questo esempio viene indicato che il vettore $\overrightarrow{U}$ ha un modulo dal valore 3 e un inclinazione con gli assi di $30^\circ$. \\ 
\subsubsection{Versori}
Prima di illustrare la rappresentazione per versori è opportuno esprimere il concetto di versore. \\
Il versore è un normale vettore che però rispetta queste due caratteristiche :
\begin{itemize}
\item Ha modulo pari ad 1
\item Corrisponde con uno degli assi cartesiano
\end{itemize}
Introdotte queste due caratteristiche si può subito notare l'esistenza di tre versori, corrispondenti agli assi x,y,z. Questi versori sono :
\begin{itemize}
\item $\overrightarrow{i} = (1,0,0)$
\item $\overrightarrow{j} = (0,1,0) $
\item $\overrightarrow{k} =(0,0,1),$
\end{itemize}
Costruire un qualunque vettore con l'utilizzo di questi tre versori è molto semplice in quanto basterà esprimere il vettore come una sommatori di multipli di versori, ad esempio :
$$\overrightarrow{U} = (7,-3.5 ) = 7\overrightarrow{i} + (-3)\overrightarrow{j} + 5\overrightarrow{k} =(7,0,0) + (0,-3,0) + (0,0,5 ) = (7,-3,5)$$ \\ 
\subsection{Prodotto scalare}
Un' operazione molto importante per i vettori è il prodotto scalare (il cui metodo di "elaborazione", per la rappresentazione in coordinate,è illustrato nella sezione del 04/03/2015 alla voce "operazione sui vettori") che quando siamo in rappresentazione per modulo è argomento funziona cosi (presi $\overrightarrow{U} = (\rho_1,\theta_1)$ e $\overrightarrow{V} = (\rho_2,\theta_2)$) :
$$\overrightarrow{U} * \overrightarrow{V} = |\rho_1||\rho_2| * \cos(\theta_1 - \theta_2)$$ \\ 
Da quest'ultima formula si nota subito una delle cose fondamentali del prodotto scalare, ovvero che esprime al suo interno l'angolo che si forma tra i due vettori, espresso sotto la funzione cos. Ed è proprio dalla funzione cos che si ricava subito un'informazione importante sul prodotto scalare, ovvero che se due vettori sono ortogonali (ovvero formano un angolo di 90 gradi) il loro prodotto scalare sarà uguale a zero, in quanto il coseno di 90 gradi corrisponde a 0. \\
Ovviamente nel caso di di vettori espressi in modulo e argomento è inutile tutto questo discorso in quanto per calcolare l'angolo compreso basterà fare la differenza tra gli angoli dei due vettori, ma nel caso in cui invece fossimo nella rappresentazione per coordinate tutto può diventare estremamente utile, e in quel caso la formula del prodotto scalare (nota : ciò vale solo nei reali ) corrisponde a questa :
$$\overrightarrow{U} * \overrightarrow{V} = |\overrightarrow{U}||\overrightarrow{V}| * \cos(\widehat{UV})$$
Da cui è possibile ricavare l'angolo in questo modo :
$$\widehat{UV} = \arccos \left( \frac{ |\overrightarrow{U}||\overrightarrow{V}|}{\overrightarrow{U} * \overrightarrow{V} } \right)$$ \\\
\subsection{Cinematica}
\hspace*{1cm}
\subsubsection{Traiettoria}
La \textbf{traiettoria} è una funzione nello spazio \underline{f(x,y,z) = 0} che indica tutti i punti dello spazio che sono stati, sono e saranno percorsi da un corpo durante il suo movimento. Ho usato il passato, il presente e il futuro contemporaneamente perché noi non sappiamo il tempo necessario a raggiungere un punto o ad effettuare uno spostamento, sempre se il corpo si muovo solo in un verso e che non faccia avanti e indietro, cosa che può accadere, ma sappiamo solo che il corpo durante il suo movimento passerà dal quel punto.
\subsubsection{Equazione oraria}
Se invece desiderassimo mettere in relazione la posizione di un corpo coni il tempo allora dovranno sarà necessaria un'equazione oraria, ovvero appunto un equazione che mette il relazione il punto dove si trova il corpo con il tempo e si indica nel seguente modo (per i tre assi cartesiani ) :
\begin{itemize}
\item $x(t) = ....$
\item $y(t) = ....$
\item $z(t) = ....$
\end{itemize}
Dove nel termine a destra appare appunto la posizione su uno degli assi in funzione del tempo, mentre a sinistra apparirà un normale membro di un'equazione. \\
Detto questo si può indicare come le coordinate di un punto in funzione del tempo siano : 
$$\overrightarrow{P} = (x(t),y(t),z(t))$$ 
\subsubsection{Velocità e accelerazione}
\hspace*{1cm} \\
\textbf{\textit{Velocità}} \\ \\ 
Data un'equazione oraria, viene definita come \textbf{velocità} la derivata dell'equazione oraria in funzione del tempo.
Molto importante osservare che la velocità non è un semplice scalare ma bensì un vettore formato da un modulo, che indica la "quantità di velocità" ed una direzione, che indica la direzione verso la quale si sposterà il corpo. Vedendo la velocità come un vettore, è importante constatare che le componenti che compongono il vettore sono appunto gli spostamenti che tale velocità produce lungo gli assi cartesiani (quindi avrà una componente x, una y e una z). \\ \\
\textbf{Importante: }La velocità è un vettore che è sempre parallelo alla traiettoria del corpo. \\ \\
\textbf{\textit{Velocità Istantanea}} \\ \\ 
La \textbf{velocità istantanea} di un corpo viene definito come 
$$\lim_{t \to 0} \frac{P_2(x_2(t),y_2(t),z_2(t)) - P_1(x_1(t),y_1(t),z_1(t))}{\Delta \; t}$$
Con $P_1 \; e \; P_2$ due punti lungo la traiettoria del corpo. \\ 
La velocità istantanea è per definizione sempre (in qualunque punto) \textbf{tangente} alla traiettoria nel punto nel quale vogliamo calcolare la velocità istantanea. \newpage
\textbf{\textit{Accelerazione}} \\ \\ 
L'accelerazione viene definita come la \textbf{variazione di velocità} nel tempo ed è quindi la derivata prima della velocità del tempo, e di conseguenza la derivata seconda dell'equazione oraria, ad esempio :
$$\overrightarrow{a} = \frac{d\overrightarrow{V}}{dx} = \frac{d^2 \overrightarrow{P}}{dx}$$
Ovviamente anche l'accelerazione è un vettore che avrà come componenti : 
\begin{itemize}
\item Come  x avrà la derivata della componente x del vettore velocità
\item Come  y avrà la derivata della componente y del vettore velocità
\item Come  z avrà la derivata della componente z del vettore velocità
\end{itemize}
Per visualizzare questo meglio 
$$\overrightarrow{a} = (\frac{dV_x(t)}{dt},\frac{dV_y(t)}{dt},\frac{dV_z(t)}{dt})$$
\subsubsection{Esempi}
Supponiamo di avere l'equazione oraria $x(t) = A\cos(\omega t)$, trovare $\overrightarrow{V}$ e $\overrightarrow{a}$ : \\ 
Calcoliamo la velocità effettuando la derivata dell'equazione oraria :
$$V(t) = \frac{d}{dx}x(t)= \frac{d}{dx} A\cos(\omega t) = -\omega A \sin(\omega t)$$
E dalla velocità calcoliamo l'accelerazione(sempre in funzione del tempo):
$$a(t) = \frac{d}{dx}V(t) = \frac{d}{dx} -\omega A \sin(\omega t) = -\omega^2 A \cos (\omega t)$$
Notando il risultato salta subito all'occhio che l'accelerazione del corpo dipende dalla posizione del corpo stesso infatti :
$$a(t) = -\omega^2 A\cos (\omega t) =  -w^2x(t)$$
Quando abbiamo che l'accelerazione di un corpo dipende dalla sua posizione si dice che abbiamo un \textbf{moto armonico}. \\
Più in generale se $\frac{d^2 f(t)}{dt^2} = -Kf(t) $ si ha un moto periodico (f(t) non importa che sia obbligatoriamente x(t), ma può essere una qualunque funzione nel tempo). \newpage
\section{Appunti 05/03/2015}
\subsection{Osservazione}
Come si può osservare da questa equazione oraria
$$x(t) = 2t^2 + 3t^3$$
Applicando l'operazione di derivata viene ridotto il grado dell'equazione di un'unità per derivazione ( pura e semplice matematica ) che ci permettono di fare alcune considerazioni :
\begin{enumerate}
\item Se un'equazione oraria è di grado 3 a(t) lineare
\item Se un'equazione oraria è di grado 2 a(t) costante e v(t) lineare (uniformemente accelerata lungo quell'asse)
\item Se un'equazione oraria è di grado 1 a(t) = 0 e v(t) costante (ovviamente lungo l'asse dell'equazione)
\end{enumerate}
Infatti tornando all'equazione di prima notiamo che $v(t) = 4t + 9t^2$ e $a(t) = 4 + 18t$ (che è  lineare).
\subsection{Esercizi}
\subsubsection{Primo esercizio}
Prendiamo un sistema di due equazioni orarie (che descrivono quindi lo spostamento di un corpo lungo due assi)
\[
\begin{array}{lr}
	x(t) = 2t^2 -3t^3 \\ 
	y(t) = 5t + 4 \\
\end{array}
\]
trovare l'angolo tra vettore velocità e vettore accelerazione all'istante 1. \\ 
Innanzi tutto procediamo a trovare i vettori velocità e accelerazione per entrambi gli assi :
\[
\begin{array}{lr}
	V_x(t) = 4t -9t^2  \qquad \Longrightarrow a_x(t) = 4 - 18t\\ 
	V_y(t) = 5  \qquad \quad \qquad \Longrightarrow a_y(t) = 0
\end{array}
\]
Adesso non ci rimane che sostituire 1 a t per ottenere il vettore velocità e accelerazione :
\[
\begin{array}{lr}
	V_x(1) = 4(1) -9(1)^2 = 4 - 9 = -5  \quad \Longrightarrow a_x(1) = 4 - 18(1) = 4- 18 = -14\\ 
	V_y(1) = 5  \qquad \qquad \quad \qquad \qquad \quad \qquad \Longrightarrow a_y(1) = 0
\end{array}
\]
Ecco che abbiamo ottenuto che V(1) = (1,1) e a(1) = (-14,0). \\  
Per finire non ci rimane che ricordarsi che il prodotto scalare contiene il coseno dell'angolo compreso e applicare la formula per trovare un angolo a partire dal prodotto scalare :
$$\widehat{V(1)a(1)} = \arccos \left(\frac{V(1) a(1)}{|V(1)||a(1)|} \right) =\arccos \left(\frac{70}{14\sqrt{50}} \right) = \arccos \left( \frac{1}{\sqrt{2}} \right) = 45 ^ \circ$$ 
\subsubsection{Secondo esercizio}
Calcolare la traiettoria dato il seguente sistema di equazioni orarie :
\[
\begin{array}{lr}
	x(t) = 6t^2 + 3  \Longrightarrow x = 6\frac{y}{2} + 3 \Longrightarrow 2x- 3y^2 - 4 = 0\\
	y(t) = 2t  \; \qquad \Longrightarrow t = \frac{y}{2}
\end{array}
\]
Dalla seconda equazione ricavo t (che poi andrà sostituito per farlo sparire, perché un equazione oraria non presenta il tempo) e lo sostituisco, ricavando l'equazione oraria che poi verrà sistemata con semplici passaggi algebrici(equazione oraria ottenuta : $ 2x- 3y^2 - 4 = 0$ ).
\subsubsection{Terzo esercizio}
Dato un sistema di equazioni orarie lungo due assi trovare, velocità,accelerazione e traiettoria:
\[
\begin{array}{lr}
	x(t) = A\cos(\omega t) \Longrightarrow V_x(t) = -\omega A \sin(\omega t) \Longrightarrow 
				a_x(t) = -\omega^2 A^2\cos(\omega t)\\
	y(t) = A\sin(\omega t) \, \Longrightarrow V_y(t) = \omega A\cos(\omega t) \; \;\,\Longrightarrow a_y(t) = -\omega^2 A^2 \sin(\omega t)
\end{array}
\]
A questo punto abbiamo trovato velocità e accelerazione, non ci rimane che trovare la traiettoria, ma questa volta non è possibile sostituire come prima, ma visto che è presente il seno e il coseno possiamo elevare tutto al quadrato e usare le proprietà di seno e coseno.
\[
\begin{array}{lr}
	x^2(t) = A^2\cos^2(\omega t) \Rightarrow x^2(t) = A^2(1 - \sin^2(\omega t)) \Rightarrow x^2 = A^2 (1 -\frac{y^2}{A^2})\Rightarrow x^2 + y^2 = A^2\\
	y(^2t) = A^2\sin^2(\omega t)  \Rightarrow \sin^2(\omega t) = \frac{y^2}{A^2}
\end{array}
\]
Dall'equazione oraria($x^2 + y^2 = A^2$) si vede che il corpo si muove secondo una traiettoria circolare (moto circolare) di raggio A. \\
Essendo questo un moto curvilineo siamo sicuri che la velocità non sia costante, perché anche se della velocità ne è costante il modulo, non lo sarà sicuramente la direzione (che è tangente alla traiettoria),calcoliamo quindi il modulo della velocità :
$$|\overrightarrow{V}| = \sqrt{V_x ^2 + V_y ^ 2} = \sqrt{\omega^2 A^2 \sin^2(\omega t) + \omega^2 A^2 \cos^2(\omega t)} = \omega A\$$
Dato che A è il raggio si può constatare che il  modulo della velocità del corpo dipende dal raggio della sua traiettoria. \\
Adesso calcoliamo il modulo dell'accelerazione:
$$|\overrightarrow{a} = \sqrt{}$$
\section{Appunti del 06/03/2015}
\end{document}