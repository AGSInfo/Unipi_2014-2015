\documentclass[fontsize = 20px, paper = a4]{article}
\author{Alessandro Sieni, Gianluca Mondini}
\title{Appunti Fisica generale}
\date{\today}
\usepackage{amsfonts}
\usepackage[utf8]{inputenc}
\begin{document}
\maketitle
\newpage
\tableofcontents
\newpage
\section{Appunti 03/03/2015}
\subsection{Ordini di grandezza}
Quando si parla di ordini di grandezza intendiamo non una precisa quantità ma un'indicazione utile ad effettuare delle stime che non devono essere necessariamente precise.
\subsection{Calcolo dimensionale}
\subsubsection{Definizione}
Il calcolo dimensionale ci permette di lavorare solo con le dimensioni che compongono le componenti da studiare e ci permette di verificare se alcuni procedimenti o calcoli sono corretti dal punto di vista dimensionale, ovvero se la dimensione ottenuta è coerente con quella di ciò che dobbiamo calcolare.
Nel caso di alcuni casi semplici è possibile ottenere delle formule senza alcun procedimento, ma solo mediante l'utilizzo del caso dimensionale.
\subsubsection{Esempi}
Esaminiamo in questo esempio la formula per ottenere il tempo di caduta di un corpo da una fissata altezza.
Solo studiando in maniera puramente intuitiva il fenomeno una persona può ipotizzare che la velocità possa dipendere dall'accelerazione di gravita( g ) dal tempo di caduta (t) e dall'altezza dalla quale il corpo viene lanciato, quindi riportando queste tre grandezze sotto forma di dimensioni otteniamo che:
$$g = [V][T]^-1 = [L][T]^-2$$
e dato che $[V] = [L][T]^-1$ dobbiamo riottenere la stessa dimensione, e per l farlo dobbiamo trovare gli esponenti $x,y,z$ da assegnare a g,t,h che si andranno a moltiplicare, che ci permettano di ritrovare la stessa dimensione, in poche parole :
$$[L][T]^-1 = \underbrace{([L][T]^-2)^x}_g \underbrace{[T]^y}_t \underbrace{[L]^z}_h$$
Da questo procediamo in modo algebrico ottenendo
$$[L][T]^-1 = \underbrace{[L]^x[T]^{-2x}}_g \underbrace{[T]^y}_t \underbrace{[L]^z}_h$$
da cui abbiamo un sistema a due equazioni ma a tre variabili (x,y,z) in quanto
$$
\begin{array}{lr}
   x+z = 1 \\
   y - 2x = -1
  \end{array}
$$
Per risolvere questo problema proviamo a non considerare l'altezza come fattore determinante per calcolare la velocità e troviamo quindi come si comporta il tempo di caduta rispetto a g e h.
$$[T] = [T]^y ([L][T]^-2)^x$$
ottenendo stavolta un sistema a due equazioni e due incognite che ha come soluzione $x = -\frac{1}{2}$ e $y = \frac{1}{2}$. Infatti la formula del tempo di caduta di un corpo è proprio $\sqrt{\frac{h}{g}}$-. \\
\subsection{Vettori}
\hspace*{0.1cm}
\subsubsection{Introduzione}
\hspace*{0.1cm} \\ 
In fisica l'utilizzo di vettori invece che di semplici scalari è fondamentale in quanto ci permettono di rappresentare in modo corretto la realtà (ad esempio non è possibile rappresentare la posizione di un corpo con un solo numero, ma ne sono necessari 3, uno per ogni asse cartesiano x,y,z).
\subsubsection{Operazioni sui vettori}
Le operazioni che è possibile effettuare sui vettori sono le seguenti :
\begin{itemize}
\item Somma
\item Prodotto tra un vettore ed uno scalare
\item Prodotto scalare tra due vettori
\item Prodotto vettoriale
\end{itemize}
\textbf{\textit{Somma}} \\  \\
La somma tra due vettori restituisce un vettore le cui componenti corrispondono alla sommatoria delle componenti relativi alla solita posizione, esempio :
$$(1,2,3) + (7,-3,4) = (1+ 7,2 + (-3),3+4) = (8,-1.7)$$ \\  
\textbf{\textit{Prodotto di un vettore per uno scalare}} \\ \\
Il prodotto di un vettore per uno scalare si ottiene moltiplicando ciascuna componente per lo scalare, esempio :
$$3 * (4,2,5) = (4 * 3, 2 * 3, 5* 3) = (12,6,15)$$ \\ 
\textbf{\textit{Prodotto scalare tra due vettori}} \\ \\
Il risultato del prodotto scalare tra due vettori sarà appunto uno scalare che corrisponderà alla sommatoria del prodotto delle relative componenti dei due vettori, esempio presi $U = (1,2,3)$ e $V=(3,4,5)$ il prodotto scalare tra U e V sarà :
$$U*V = \sum_{i = 1}^3 (U_i * V_I) = (1* 3) + (2*4) + (5 * 3) = 26$$ \\ 
\textbf{\textit{Prodotto vettoriale tra due vettori}} \\ \\
\subsubsection{Rappresentazione di un vettore}
\hspace*{5cm} \\
Per rappresentare un vettore ci sono 4 possibili modi : 
\begin{itemize}
\item Mediante le coordinate
\item Mediante una freccia nel piano
\item Mediante il modulo del vettore e il suo argomento
\item Mediante l'utilizzo di versori
\end{itemize}
\hspace*{1cm}\\
 \textbf{\textit{Coordinate}} \\  \\
Quando si rappresenta un vettore mediante coordinate si esplicita una tupla di n numeri l'esempio più classico è $V = (1,2,3)$ \\ \\ 
\textbf{\textit{Freccia nel Piano}} \\  \\
Quando si rappresenta un vettore con una freccia si utilizzano le stesse coordinate illustrate sopra ma le si proiettano in un ipotetico spazio n-dimensionale (ovviamente questo nei modelli matematici, perché in quelli fisici non si supera la terza dimensione, eccetto alcuni rari casi) e si traccia una freccia che parte dall'origine degli assi e che arriva proprio nel punto dello spazio avente come coordinate cartesiane le componenti del vettore. \\ \\
\textbf{\textit{Modulo e Argomento}} \\  \\
\section{Appunti 04/03/2015}
\section{Appunti 05/03/2015}
\end{document}