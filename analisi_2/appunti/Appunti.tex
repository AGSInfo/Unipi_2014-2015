\documentclass[fontsize = 20px, paper = a4]{article}
\author{Alessandro Sieni, Gianluca Mondini}
\title{Appunti Analisi II }
\date{\today}
\usepackage{amsfonts}
\usepackage[utf8]{inputenc}


\begin{document}
\maketitle
\newpage
\tableofcontents
\newpage
\section{Appunti 03/03/2015}
\subsection{Definizioni}
\subsubsection{Convergenza}
\hspace*{1cm}
$$\forall \; \epsilon > 0 \quad \exists \; \nu \; : \; L-\epsilon<x_n<L+\epsilon \quad \forall \; n > \nu$$ 
\subsubsection{Sfera}
\hspace*{1cm}
$$B_\delta(x_0) = \left \{ y \in X \; : \; d(y,x_0) < \delta \right\}$$ \\
Una sfera è un insieme di elementi che distano dal centro della sfera (indicato dall'elemento tra parentesi) di massima distanza uguale al raggio (indicato dal pedice sulla lettera B).\\
Nel caso in cui (come quella sopra) la distanza sia obbligatoriamente \textbf{minore} del raggio allora la sfera si dice \textbf{aperta}, altrimenti se la distanza è minore o uguale del raggio la sfera si dice \textbf{chiusa}. \\
Esempio di sfera chiusa:
$$B_\delta(x_0) = \left \{ y \in X \; : \; d(y,x_0) \le \delta \right\}$$ 
Quando siamo in uno spazio metrico la sfera viene definita anche intorno.
\subsubsection{Punto intero}
Un punto $x_0$si dice interno se preso un insieme $\Omega \subseteq \mathbb{R}^n$ si verifica la seguente condizione : 
$$\exists \; \delta > 0 : B_\delta (x_0) \subseteq \Omega$$
Ovvero spiegato a parole se l'insieme $\Omega$ contiene il punto (ovvero il centro della sfera) ed anche tutta la sfera di raggio $\delta$ a piacere.
\subsubsection{Punto esterno}
Un punto $x_0$ si dice esterno ad un insieme $\Omega \in \mathbb{R}^n$ se :
$$\exists \; \delta > 0 \; : \; B_\delta (x_0) \cap \Omega = 0$$
Ovvero un punto si dice esterno ad un insieme se una qualunque sfera centrata nel punto di raggio a piacere non si interseca con l'insieme.
\subsubsection{Punto di frontiera}
Un punto $x_0$si dice di frontiera se preso un insieme $\Omega \subseteq \mathbb{R}^n$ si verifica la seguente condizione : 
$$\forall \; \delta > 0 \; \exists \; x_1,x_2 \in B_\delta (x_0) \; : \; x_1 \in \Omega \; e \; x_2 \ni\Omega $$
Ovvero un punto si dice di frontiera se sta al "bordo"dell'insieme, ovvero se presa una qualunque sfera centrata nel punto $x_0$ ci sarà almeno un punto appartenente alla sfera interno all'insieme e uno esterno all'insieme.
\section{Appunti 04/03/2015}
\subsection{Diametro di un insieme}
\subsubsection{Definizione}
Il diametro di un insieme $\Omega$ è definito come :
$$\sup_{x,y \in \Omega|}|x-y|$$
\subsubsection{Esempi}
Consideriamoci in $\mathbb{R}^2$ e più precisamente consideriamo $\Omega$ come il cerchio unitario, quindi da ciò ne deriva che presi $x,y \in \Omega$, $|x| \le 1 \; e \; |y| \le 1$ quindi $ | x - y | \le 2$ e dalla disuguaglianza triangolare ( che ricordiamo dice $|x+y| \le |x| + |y|$ ): 
$$|x-y| \le |x| + |-y| \Longrightarrow |x-y| \le |x| + |y| \le 1 +1 = 2$$  \\ 
Quindi considerando la sfera $B_\delta(x_0)$ e presi $x = (0,1)$ e $y=(-1,0)$ otteniamo che:
$$sup_{B_\delta(x_0)} |x-y| \ge |(1,0) - (-1,0)| = |(2,0)| = 2$$ \\ 
La disuguaglianza sopra deriva dal fatto che l'estremo superiore di $|x-y|$ è sempre maggiore uguale di $|x-y|$ qualunque punto si scelga, come ad esempio i punti $x = (0,1)$ e $y = (-1,0)$.
\emph{Altro esempio}
Scelto \\
$$\Omega = \left \{ x\in \mathbb{R}^2 \; : \; |x| < 1 \right \} \quad \Longrightarrow \quad sfera \; aperta $$ \\
Abbiamo già provato prima che 2 è un maggiorante e dalla definizione di diametro otteniamo che:
$$d\;\Omega \le 2$$ \\
In quanto definito la sfera come un insieme aperto e quindi di conseguenza $|y - x| = 2 -2\epsilon$. \\
Il $2\epsilon$ deriva dal fatto che essendo aperta la sfera  (ovvero che $|x| < 1$ ) il grande valore che può assumere sarà 1 - una piccolissima, quanto si vuole, quantità che noi chiameremo $\epsilon$ che gli impedirà di raggiungere 1. Ciò avviene anche quando x tenta di raggiungere il valore di -1 che in valore assoluto corrisponde ad 1, quindi posto $x = 1-\epsilon$ e $y = -1 + \epsilon$ si ottiene $1 - \epsilon - (-1 + \epsilon) = 2 - 2\epsilon$. \\ 
\subsubsection{Teorema}
\textbf{Teorema: }\emph{Se un'insieme $\Omega$ è limitato $\Longleftrightarrow$ $d \; \Omega < +\infty$ }\\ \\ \underline{DIMOSTRAZIONE} \\ \\ 

\subsection{Convergenza di successioni}
\subsubsection{Lemma}
\textbf{Lemma: } \emph{Sia} $\quad x \in \mathbb{R}^N \quad \Longrightarrow x = (x_1,x_2,......,x_N)$ \\\\ 
\textbf{Allora: } \hspace{0.2cm}$|x_i| \le |x| \quad \forall \; i = 1....N $ \\ \\
\underline{DIMOSTRAZIONE} \\ \\
Consideriamo
$$|x| \le \sqrt{N} * \max_{i = 1...N}(|x_i|)$$ \\
Da cui otteniamo sostituendo alla norma del vettore il modo con cui è possibile calcolarla
$$\sqrt{\sum_{i = 1}^n (x_i)^2} \le \sqrt{N} * \max_{i = 1...N}(|x_i|)$$ \\
A questo punto effettuiamo una stima dall'alto considerando che :
$$\sum_{i = 1} ^ n (x_i)^2 \ge \sqrt{x_i ^2}= |x_i| \quad \forall \; i = 1....n$$
Mentre per effettuare una stima dal basso poniamo la seguente equazione:
$$\sum_{i = 1}^n x_i ^2 \le N * \max_{i = 1...n}(x_i)^2$$ \\
Adesso possiamo applicare la radice a entrambi i membri perché non varia il segno della disequazione  e otteniamo:
$$\sqrt{\sum_{i = 1}^n x_i ^2} \le \sqrt{ N * (\max_{i = 1...n}(x_i)) ^ 2}$$ \\
Ma dato che $(\max_{i = 1...n}(x_i))^2$ è uguale ad $\max_{i = 1...n}(x_i)^2$ quindi anche $\sqrt{(\max_{i = 1...n}(x_i))^2}$ è uguale a $\sqrt{\max_{i = 1...n}(x_i)^2}$ che a sua volta corrisponde a $\max_{i = 1...n}|(x_i)|$, portando anche a termine la stima dal basso e quindi la dimostrazione.
\subsubsection{Definizione di convergenza}
La definizione di convergenza è: 
$$\forall \; \epsilon > 0 \quad \exists \; \delta \; : \; \forall \; n > \delta \; \; |x_n - \delta| < \epsilon$$ \\ \\
In poche parole la convergenza di una successione vettoriale (ovvero composta da vettori e non scalari) convergerà al punto le cui componenti corrisponderanno al punto di convergenza  di ogni singola componente.
\subsubsection{Esempio}
Un chiaro esempio della definizione è :
$$\left( \frac{\sin\left(\frac{1}{n}\right)}{\frac{1}{n}},\frac{1}{n}\right) \longrightarrow \left(1,0\right)$$ \\ 
Questo risultato è dato dal fatto che 
$$\lim_{n \to \infty} \frac{\sin\left(\frac{1}{n}\right)}{\frac{1}{n}} = 1 \quad e \quad \lim_{n \to \infty}\frac{1}{n} = 0$$



\end{document}
