%% LyX 2.1.2.2 created this file.  For more info, see http://www.lyx.org/.
%% Do not edit unless you really know what you are doing.
\documentclass[fontsize=20px,paper=a4]{article}
\usepackage[utf8]{inputenc}

\makeatletter
%%%%%%%%%%%%%%%%%%%%%%%%%%%%%% User specified LaTeX commands.
\author{Alessandro Sieni, Gianluca Mondini}
\title{Appunti Analisi II }
\date{\today}
\usepackage{amsfonts}
<<<<<<< HEAD
\usepackage[utf8]{inputenc}
=======





\makeatother

>>>>>>> origin/master
\begin{document}
\maketitle \newpage{}\tableofcontents{}\newpage{}


\section{Appunti 03/03/2015}


\subsection{Definizioni}


\subsubsection{Convergenza}

\hspace*{1cm} 
\[
\forall\;\epsilon>0\quad\exists\;\nu\;:\; L-\epsilon<x_{n}<L+\epsilon\quad\forall\; n>\nu
\]



\subsubsection{Sfera}

\hspace*{1cm} 
\[
B_{\delta}(x_{0})=\left\{ y\in X\;:\; d(y,x_{0})<\delta\right\} 
\]
\\
 Una sfera è un insieme di elementi che distano dal centro della sfera
(indicato dall'elemento tra parentesi) di massima distanza uguale
al raggio (indicato dal pedice sulla lettera B).\\
 Nel caso in cui (come quella sopra) la distanza sia obbligatoriamente
\textbf{minore} del raggio allora la sfera si dice \textbf{aperta},
altrimenti se la distanza è minore o uguale del raggio la sfera si
dice \textbf{chiusa}. \\
 Esempio di sfera chiusa: 
\[
B_{\delta}(x_{0})=\left\{ y\in X\;:\; d(y,x_{0})\le\delta\right\} 
\]
Quando siamo in uno spazio metrico la sfera viene definita anche intorno. 


\subsubsection{Punto intero}

Un punto $x_{0}$si dice interno se preso un insieme $\Omega\subseteq\mathbb{R}^{n}$
si verifica la seguente condizione : 
\[
\exists\;\delta>0:B_{\delta}(x_{0})\subseteq\Omega
\]
Ovvero spiegato a parole se l'insieme $\Omega$ contiene il punto
(ovvero il centro della sfera) ed anche tutta la sfera di raggio $\delta$
a piacere. 


\subsubsection{Punto esterno}
<<<<<<< HEAD
Un punto $x_0$ si dice esterno ad un insieme $\Omega \in \mathbb{R}^n$ se :
$$\exists \; \delta > 0 \; : \; B_\delta (x_0) \cap \Omega = 0$$
Ovvero un punto si dice esterno ad un insieme se una qualunque sfera centrata nel punto $x_0$ di raggio a piacere non si interseca con l'insieme.
\subsubsection{Punto di frontiera}
Un punto $x_0$si dice di frontiera se preso un insieme $\Omega \subseteq \mathbb{R}^n$ si verifica la seguente condizione : 
$$\forall \; \delta > 0 \; \exists \; x_1,x_2 \in B_\delta (x_0) \; : \; x_1 \in \Omega \; e \; x_2 \ni\Omega $$
Ovvero un punto si dice di frontiera se sta al "bordo"dell'insieme, ovvero se presa una qualunque sfera centrata nel punto $x_0$ ci sarà almeno un punto appartenente alla sfera interno all'insieme e uno esterno all'insieme.
\subsubsection{Insieme aperto/chiuso}
Un insieme si dice \textbf{aperto} se ogni suo punto è interno (non contenendo quindi alcun punto di frontiera).\\
Un insieme si dice \textbf{chiuso} se contiene la propria frontiera.
\subsubsection{Punto di accumulazione}
Un punto $x_0$ si dice di accumulazione di un insieme $\Omega$ (si scrive $x_0 \in \partial \; \Omega$ se :
$$\forall \; \delta > 0 \quad \exists\;x \in \; B_\delta (x_0) \; \cap \;\Omega \quad x \neq x_0$$ 
\subsubsection{Punto isolato}
Un punto $x_0$ si dice isolato rispetto ad un insieme $\Omega$ se :
$$\exists \; \delta \; : \; \Omega \; \cap \; B_\delta (x_0) = x_0$$ \\
Un esempio di punto isolato può essere un generico punto $x_0$ appartenente ad $\Omega$, che viene definito nel seguente mod: $\Omega = \Omega' + \{x_0\}$ con $\Omega'$ e $\{x_0 \}$ molto distanti.
\subsubsection{Insieme limitato}
Un insieme $\Omega $ si dice limitato se 
$$\exists \; [H,K] \sup	seteq \Omega$$ 
Quindi se l'insieme $\Omega$ è contenuto in un intervallo chiuso i cui estremi sono H e K. \\
Un altro modo per definire un'insieme limitato è:
$$\exists \; x_0, \delta \;:\; \Omega \subseteq B_\delta (x_0)$$
Ovvero un insieme si dice limitato se esiste un punto $x_0$ la cui sfera di raggio $\delta$ di dimensione a piacere contiene tutto l'insieme. Dato che una sfera è limitata se contiene tutto l'insieme anche quest'ultimo per forza di cose dovrà essere limitato.
=======

Un punto $x_{0}$ si dice esterno ad un insieme $\Omega\in\mathbb{R}^{n}$
se : 
\[
\exists\;\delta>0\;:\; B_{\delta}(x_{0})\cap\Omega=0
\]
Ovvero un punto si dice esterno ad un insieme se una qualunque sfera
centrata nel punto di raggio a piacere non si interseca con l'insieme. 


\subsubsection{Punto di frontiera}

Un punto $x_{0}$si dice di frontiera se preso un insieme $\Omega\subseteq\mathbb{R}^{n}$
si verifica la seguente condizione : 
\[
\forall\;\delta>0\;\exists\; x_{1},x_{2}\in B_{\delta}(x_{0})\;:\; x_{1}\in\Omega\; e\; x_{2}\ni\Omega
\]
Ovvero un punto si dice di frontiera se sta al \textquotedbl{}bordo\textquotedbl{}dell'insieme,
ovvero se presa una qualunque sfera centrata nel punto $x_{0}$ ci
sarà almeno un punto appartenente alla sfera interno all'insieme e
uno esterno all'insieme. 


>>>>>>> origin/master
\section{Appunti 04/03/2015}


\subsection{Diametro di un insieme}


\subsubsection{Definizione}

Il diametro di un insieme $\Omega$ è definito come : 
\[
\sup_{x,y\in\Omega|}|x-y|
\]



\subsubsection{Esempi}

Consideriamoci in $\mathbb{R}^{2}$ e più precisamente consideriamo
$\Omega$ come il cerchio unitario, quindi da ciò ne deriva che presi
$x,y\in\Omega$, $|x|\le1\; e\;|y|\le1$ quindi $|x-y|\le2$ e dalla
disuguaglianza triangolare ( che ricordiamo dice $|x+y|\le|x|+|y|$
): 
\[
|x-y|\le|x|+|-y|\Longrightarrow|x-y|\le|x|+|y|\le1+1=2
\]
\\
 Quindi considerando la sfera $B_{\delta}(x_{0})$ e presi $x=(0,1)$
e $y=(-1,0)$ otteniamo che: 
\[
sup_{B_{\delta}(x_{0})}|x-y|\ge|(1,0)-(-1,0)|=|(2,0)|=2
\]
\\
 La disuguaglianza sopra deriva dal fatto che l'estremo superiore
di $|x-y|$ è sempre maggiore uguale di $|x-y|$ qualunque punto si
scelga, come ad esempio i punti $x=(0,1)$ e $y=(-1,0)$. \emph{Altro
esempio} Scelto \\
 
\[
\Omega=\left\{ x\in\mathbb{R}^{2}\;:\;|x|<1\right\} \quad\Longrightarrow\quad sfera\; aperta
\]
\\
 Abbiamo già provato prima che 2 è un maggiorante e dalla definizione
di diametro otteniamo che: 
\[
d\;\Omega\le2
\]
\\
 In quanto definito la sfera come un insieme aperto e quindi di conseguenza
$|y-x|=2-2\epsilon$. \\
 Il $2\epsilon$ deriva dal fatto che essendo aperta la sfera (ovvero
che $|x|<1$ ) il grande valore che può assumere sarà 1 - una piccolissima,
quanto si vuole, quantità che noi chiameremo $\epsilon$ che gli impedirà
di raggiungere 1. Ciò avviene anche quando x tenta di raggiungere
il valore di -1 che in valore assoluto corrisponde ad 1, quindi posto
$x=1-\epsilon$ e $y=-1+\epsilon$ si ottiene $1-\epsilon-(-1+\epsilon)=2-2\epsilon$.
\\
 


\subsubsection{Teorema}

\textbf{Teorema: }\emph{Se un'insieme $\Omega$ è limitato $\Longleftrightarrow$
$d\;\Omega<+\infty$ }\\
 \\
 \underline{DIMOSTRAZIONE} \\
 \\



\subsection{Convergenza di successioni}


\subsubsection{Lemma}


\subsubsection{\emph{Sia} $\quad x\in\mathbb{R}^{N}\quad\protect\Longrightarrow x=(x_{1},x_{2},......,x_{N})$
\protect \\
 Allora:  \hspace{0.2cm}$|x_{i}|\le|x|\quad\forall\; i=1....N$ \protect \\
Dimostrazione}

Consideriamo 
\[
|x|\le\sqrt{N}*\max_{i=1...N}(|x_{i}|)
\]
\\
 Da cui otteniamo sostituendo alla norma del vettore il modo con cui
è possibile calcolarla 
\[
\sqrt{\sum_{i=1}^{n}(x_{i})^{2}}\le\sqrt{N}*\max_{i=1...N}(|x_{i}|)
\]
\\
 A questo punto effettuiamo una stima dall'alto considerando che :
\[
\sum_{i=1}^{n}(x_{i})^{2}\ge\sqrt{x_{i}^{2}}=|x_{i}|\quad\forall\; i=1....n
\]
Mentre per effettuare una stima dal basso poniamo la seguente equazione:
<<<<<<< HEAD
$$\sum_{i = 1}^n x_i ^2 \le N * \max_{i = 1...n}(x_i)^2$$ \\
Adesso possiamo applicare la radice a entrambi i membri perché non varia il segno della disequazione  e otteniamo:
$$\sqrt{\sum_{i = 1}^n x_i ^2} \le \sqrt{ N * (\max_{i = 1...n}(x_i)) ^ 2}$$ \\
Ma dato che $(\max_{i = 1...n}(x_i))^2$ è uguale ad $\max_{i = 1...n}(x_i)^2$ quindi anche $\sqrt{(\max_{i = 1...n}(x_i))^2}$ è uguale a $\sqrt{\max_{i = 1...n}(x_i)^2}$ che a sua volta corrisponde a $\max_{i = 1...n}|(x_i)|$, portando anche a termine la stima dal basso e quindi la dimostrazione.
\subsubsection{Definizione}
La definizione di convergenza è: 
$$\forall \; \epsilon > 0 \quad \exists \; \delta \; : \; \forall \; n > \delta \; \; |x_n - \delta| < \epsilon$$ \\ \\
In poche parole la convergenza di una successione vettoriale (ovvero composta da vettori e non scalari) convergerà al punto le cui componenti corrisponderanno al punto di convergenza  di ogni singola componente.
\subsubsection{Teorema 1}
\textbf{Teorema}: \emph{Supponiamo che l'insieme $C$ si chiuso e che la successione $x_n \in C$ \\
\hspace*{1.7cm} che $x_n \to x$} \\ \\
\textbf{Allora}: \hspace*{0.5cm}$x  \in C$ \\ \\
\underline{DIMOSTRAZIONE } \\ \\ 
Ipotizziamo per assurdo che $x \ni C$	.
Ma cosa è $x$? Iniziamo partendo dalla definizione di punto convergente:
$$\forall \; \epsilon \quad \exists \; \nu \; : \; \forall \; n > \nu \; |x_n - x| < \epsilon$$
Quindi dalla definizione si può intuire che se $|x_n - x| < \epsilon$ allora significa anche che $x_n \in B_\epsilon (x)$, ma dato che la sfera è contenuta in C si verifica un assurdo perché anche il punto interno alla sfera è contenuto in C.
\subsubsection{Teorema 2}
\textbf{Teorema: }\emph{Consideriamo il punto $x_0$ come punto di accumulazione dell'insieme \\
\hspace*{1.7cm} $\Omega$ (ovvero $x_0 \; \partial \; \Omega$) e consideriamo $x_1,x_2,....,x_N$ elementi distinti di  \\
\hspace*{1.7cm } $\Omega$} \\ \\
\textbf{Allora: } \hspace*{0.3cm}$x_n \to x_0$ \\ \\ 
\underline{DIMOSTRAZIONE} \\ \\
Per procedere con questa dimostrazione dobbiamo seguire il principio di induzione e quindi specificare


\subsubsection{Esempio}
Un chiaro esempio della definizione è :
$$\left( \frac{\sin\left(\frac{1}{n}\right)}{\frac{1}{n}},\frac{1}{n}\right) \longrightarrow \left(1,0\right)$$ \\ 
Questo risultato è dato dal fatto che 
$$\lim_{n \to \infty} \frac{\sin\left(\frac{1}{n}\right)}{\frac{1}{n}} = 1 \quad e \quad \lim_{n \to \infty}\frac{1}{n} = 0$$
\subsection{Continuità}
\subsubsection{Definizione}
Consideriamo la funzione $f \; : \; \Omega \rightarrow \mathbb{R}^N$ con $\Omega \subseteq \mathbb{R}^N$ ed un punto $x_0 \in \Omega$ una funzione si dice continua in $x_0$ se 
$$\forall \; \epsilon > 0 \quad \exists \; \delta > 0 \; : \; \forall \; x \in dom(f) \; se |x-x_0| < \delta \Rightarrow |f(x) - f(x_0)| < \epsilon$$
In questo caso la definizione è la solita per le funzioni che lavorano sui numeri reali con la differenza che però essendo su $\mathbb{R}^N$ dovrà essere calcolata la norma dei vettori e non il valore assoluto del (che comunque giusto per chiarezza corrisponde alla norma di un vettore in $\mathbb{R}$.
=======
\[
\sum_{i=1}^{n}x_{i}^{2}\le N*\max_{i=1...n}(x_{i})^{2}
\]
\\
 Adesso possiamo applicare la radice a entrambi i membri perché non
varia il segno della disequazione e otteniamo: 
\[
\sqrt{\sum_{i=1}^{n}x_{i}^{2}}\le\sqrt{N*(\max_{i=1...n}(x_{i}))^{2}}
\]
\\
 Ma dato che $(\max_{i=1...n}(x_{i}))^{2}$ è uguale ad $\max_{i=1...n}(x_{i})^{2}$
quindi anche $\sqrt{(\max_{i=1...n}(x_{i}))^{2}}$ è uguale a $\sqrt{\max_{i=1...n}(x_{i})^{2}}$
che a sua volta corrisponde a $\max_{i=1...n}|(x_{i})|$, portando
anche a termine la stima dal basso e quindi la dimostrazione. 


\subsubsection{Definizione di convergenza}

La definizione di convergenza è: 
\[
\forall\;\epsilon>0\quad\exists\;\delta\;:\;\forall\; n>\delta\;\;|x_{n}-\delta|<\epsilon
\]
\\
 \\
 In poche parole la convergenza di una successione vettoriale (ovvero
composta da vettori e non scalari) convergerà al punto le cui componenti
corrisponderanno al punto di convergenza di ogni singola componente. 


\subsubsection{Esempio}

Un chiaro esempio della definizione è : 
\[
\left(\frac{\sin\left(\frac{1}{n}\right)}{\frac{1}{n}},\frac{1}{n}\right)\longrightarrow\left(1,0\right)
\]
\\
 Questo risultato è dato dal fatto che 
\[
\lim_{n\to\infty}\frac{\sin\left(\frac{1}{n}\right)}{\frac{1}{n}}=1\quad e\quad\lim_{n\to\infty}\frac{1}{n}=0
\]



\section{Appunti 05/03/15}


\subsection{Dimostrazione del 4/3/15 rivisitata}


\subsubsection{Ipotesi}

$x_{0}\in\Game\Omega$


\subsubsection{Tesi}

$x_{n}\rightarrow x_{0}$


\subsubsection{Dimostrazione}

Si procede per induzione

$P\left(n\right)=x_{1}..x_{n}$ sono indipendenti, due a due distinti.

\[
\left|x_{i}-x_{0}\right|<\frac{1}{2^{i-1}}
\]


È necessario provare $P\left(1\right)$

$\delta=1$ $\exists\quad x_{1}$ tale che $x_{1}\in\Omega,$ $x_{1}\in B_{1}\left(x_{0}\right)$,
$x_{1}\neq x_{0}$

Supponiamo di avere $P(n)$

\[
\delta=min\left\{ \frac{1}{2^{n}},\left|x_{1}-x_{0}\right|,\left|x_{2}-x_{0}\right|...\left|x_{n}-x_{0}\right|\right\} 
\]


Questi ultimi valori sono tutti positivi maggiori di 0 in quanto l'esponenziale
è una funzione positiva, la norma è definita positiva e $x_{i}\neq x_{0}$
per ogni $i=1..n$

$\exists x_{n+1}$ tale che
\begin{enumerate}
\item $\in\Omega$
\item $\in B_{\delta}\left(x_{0}\right)$
\item $\neq x_{0}$
\end{enumerate}

\subsection{Fatto}

Se $\Omega$ è chiuso e $x_{0}\in\Game\Omega$, $x_{0}\in\Omega$


\subsection{Definizione}

Un insieme è chiuso se contiene la frontiera


\subsection{Teorema della permanenza del segno}


\subsubsection{Ipotesi}
\begin{itemize}
\item $f:\Omega\rightarrow\mathbb{R}$ continua in $x_{0}$
\item $f(x_{0})<0$
\end{itemize}

\subsubsection{Tesi}

\[
\exists\quad\delta>0\quad\forall x\in B_{\delta}(x_{0})\quad f(x)<0
\]



\subsubsection{Dimostrazione}

Dalla definizione di continuità abbiamo che: 

$\forall\quad\epsilon>0\quad\exists\quad\delta>0$ tale che $\forall x\in dom\left(f\right)$

$\left|x-x_{0}\right|<\delta$ (cioè $x\in B_{\delta}\left(x_{0}\right)$)

$\left|f(x)-f\left(x_{0}\right)\right|<\epsilon$

$f\left(x_{0}\right)-\epsilon<f(x)<f\left(x_{0}\right)+\epsilon$

siccome

$f(x)<0$ per ipotesi

Se $\epsilon<\left|f\left(x_{0}\right)\right|$ si ha $f\left(x_{0}\right)+\epsilon<0$


\subsection{Teorema di continuità della funzione somma (nome provvisorio)}


\subsubsection{Ipotesi}

$f:\Omega\rightarrow\mathbb{R}^{N}$ \\
 $g:\Omega\rightarrow\mathbb{R}^{N}$ \\
 $f,g$ continue in $x_{0}$


\subsubsection{Tesi}

$h(x)=f(x)+g(x)$ è continua in $x_{0}$ \\
 In pratica, se due funzioni sono continue in un punto anche la loro
somma sarà continua in tal punto


\subsubsection{Dimostrazione}

$\forall\epsilon>0\exists\delta>0$ tale che $\forall x\in dom(f+g)|x-x_{0}|<\delta$

\[
|h(x)-h(x_{0})|<\epsilon
\]


\[
|f(x)+g(x)-f(x_{0})-g(x_{0})|<\epsilon
\]


\[
|f(x)-f(x_{0})+g(x)-g(x_{0})|<\epsilon
\]


Grazie alla disugualianza triangolare abbiamo che:

\[
|f(x)-f(x_{0})+g(x)-g(x_{0})|<\epsilon\leq|f(x)-f(x_{0})|+|g(x)-g(x_{0})|
\]


Dall'ipotesi di continuità di f,g:\\


$\forall\sigma\exists\delta_{1}$ : $x\in dom(f):|x-x_{0}|<\delta_{1}\forall\epsilon>0|$
\\
 $|f(x)-f(x_{0})|<\sigma$ con $\sigma=\frac{\epsilon}{2}$

$\forall\sigma\exists\delta_{2}$ : $x\in dom(g):|x-x_{0}|<\delta_{2}\forall\epsilon>0|$
\\
 $|g(x)-g(x_{0})|<\sigma$ con $\sigma=\frac{\epsilon}{2}$

\[
\delta=min\lbrace\delta_{1},\delta_{2}\rbrace
\]
\[
|h(x)-h(x_{0})|<\delta
\]



\subsection{Primo teorema di composizione}


\subsubsection{Ipotesi}

$f:\Omega\rightarrow\mathbb{R}^{N}continua$ \\
 $x_{n}\in\Omega$ \\
 $x_{n}\rightarrow x\in\Omega$


\subsubsection{Tesi}

$\lim f(x_{n})=f(x)$


\subsection{Dimostrazione}

\[
\forall\epsilon>0\quad\exists\nu:\forall n>\nu\quad|f(x_{n})-f(x)|<\epsilon
\]
\[
\forall\epsilon>0\quad\exists\delta>0\quad:\quad\forall t\in\Omega|t-x|<\epsilon|f(t)-f(x)|<\epsilon
\]
Poniamo $t=x_{n}$ $|x-x_{n}|<\delta$ implica che 
\[
|f(x_{n})-f(x)|<\epsilon
\]

>>>>>>> origin/master

Per quale $n$ è verificata?
\end{document}
